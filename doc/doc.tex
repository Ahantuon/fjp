\documentclass[12pt]{report}
\usepackage[utf8]{inputenc}
\usepackage[czech]{babel}
\usepackage{hyperref}
\usepackage{graphicx} 
\usepackage{float}
\usepackage{enumitem}


\title{Překladač vlastního jazyka do instrukcí PL/0}
\author{
  Kotva, Pavel\\
  \texttt{kotva@students.zcu.cz}
  \and
  Kohlíček, Jan\\
  \texttt{kohl@students.zcu.cz}
}

\begin{document}

\begin{titlepage}
\begin{flushleft} 
{\includegraphics[width=.5\textwidth]{./img/fav_logo.jpg}\\[3cm]}
\end{flushleft}
\begin{center}

{\Large KIV/FJP - Semestrální práce}\\
\vspace{0.3cm}
{\Huge Překladač vlastního jazyka\\do instrukcí PL/0}\\
\vspace{1cm}
{\large Pavel Kotva - A17N0077P}\\
\vspace{0.1cm}
{\normalsize kotva@students.zcu.cz}\\
\vspace{0.7cm}
{\large Jan Kohlíček - A17N0075P}\\
\vspace{0.1cm}
{\normalsize kohl@students.zcu.cz}
\vfill
{\large \today}
\end{center}
\end{titlepage}

\tableofcontents
\thispagestyle{empty}

\chapter{Zadání}
Cílem práce bude vytvoření překladače zvoleného jazyka. Je možné inspirovat se jazykem PL/0, vybrat si podmnožinu nějakého existujícího jazyka nebo si navrhnout jazyk zcela vlastní. Jazyk bude překládán do instrukcí PL/0. 
\\
\\
Jazyk musí mít základní konstrukce:
\begin{itemize}[noitemsep]
	\item [-] definice celočíselných proměnných
	\item [-] definice celočíselných konstant
	\item [-] přiřazení
	\item [-] základní aritmetiku a logiku (+, -, *, /, AND, OR, negace a závorky, operátory pro porovnání čísel)
	\item [-] cyklus (libovolný)
	\item [-] jednoduchou podmínku (if bez else)
	\item [-] definice podprogramu (procedura, funkce, metoda) a jeho volání
\end{itemize}
\ \\
Rozšiřující konstrukce:
\begin{itemize}[noitemsep]
	\item [-] každý další typ cyklu
	\item [-] datový typ boolean a logické operace s ním
	\item [-] else větev
	\item [-] násobné přiřazení
	\item [-] paralelní přiřazení
	\item [-] parametry předávané hodnotou
	\item [-] návratová hodnota podprogramu
\end{itemize}

\chapter{Syntaxe}

\section{Proměnné a přiřazení}
Jazyk je staticky typový, všechny proměnné musíme nejprve deklarovat s jejich datovým typem.\\
\\
Datové typy:
\begin{description}
	\item [int] - označení celočíselného datového typu.
	\item [boolean] - logický datový typ. Boolean nabývá dvou hodnot: \textbf{true} (pravda) a \textbf{false} (nepravda). 
\end{description}
\ \\
Deklarace proměnné bez přiřazení:
\begin{verbatim}
int x;
boolean y;
\end{verbatim}
\ \\
Deklarace proměnné s přiřazení:
\begin{verbatim}
boolean x := true;
boolean y := false;
int z := (3 * 3) + 1;
\end{verbatim}
\ \\
Deklarace proměnných s násobným přiřazením:
\begin{verbatim}
int a;
int b;
int c;
a, b, c := 5;
\end{verbatim}
\ \\
Deklarace proměnných s paralelním přiřazením:
\begin{verbatim}
int a;
int b;
[a, b] := [10, 15];
\end{verbatim}





\subsection{Konstanty}
Konstanty jsou deklarovány s klíčovým slovem \textbf{const} a mohou obsahovat jen celé číslo.
\begin{verbatim}
const KONSTANTA := 5;
\end{verbatim}






\section{Podmínky (větvení)}
Podmínky zapisujeme pomocí klíčového slova \textbf{if}, za kterým následuje logický výraz. Pokud je výraz pravdivý, provede se následující příkaz. Pokud ne, následující příkaz se přeskočí a pokračuje se větví \textbf{else}. Zapsat větev \textbf{else} je povinné.
\begin{verbatim}
int i := 1;

if(i > 1)
start
    i := 2;
end
else
start
    i := 4;
end
\end{verbatim}


\subsection{Operátory}
Relační operátory, které můžeme ve výrazech používat:

\begin{table}[H]
\centering
\begin{tabular}{ c |  p{6cm}}
 Operátor & Zápis \\ \hline\hline
 	
Rovnost & \textbf{=} \\ \hline  
Je ostře větší & \textbf{\textgreater} \\ \hline
Je ostře menší & \textbf{\textless} \\ \hline
Je větší nebo rovno & \textbf{\textgreater =} \\ \hline
Je menší nebo rovno & \textbf{\textless =} \\ \hline
Nerovnost & \textbf{!=} \\ \hline
Obecná negace & \textbf{!} \\ \hline
A zároveň & \textbf{\&\&} \\ \hline
Nebo & \textbf{\textbar \textbar}  \\ \hline
\end{tabular}
\caption{Relační operátory}
\label{table:relacni_operatory}
\end{table}
\ \\
Podmínky je možné skládat a to pomocí dvou základních logických operátorů:
\begin{table}[H]
\centering
\begin{tabular}{ c |  p{6cm}}
 Operátor & Zápis \\ \hline\hline
A zároveň & \textbf{\&\&} \\ \hline
Nebo & \textbf{\textbar \textbar}  \\ \hline
\end{tabular}
\caption{Logické operátory}
\label{table:logicke_operatory}
\end{table}

\section{Cykly}


\subsection{While}
Prvním typem cyklu je \textbf{while} cyklus funguje jednoduše opakuje příkazy v bloku dokud platí podmínka. Syntaxe cyklu je následující:

\begin{verbatim}
int i := 0;

while(i < 4)
start
    i := i + 1;
end
\end{verbatim}

\subsection{Do\textendash while}
Posledním typem cyklu je \textbf{do\textendash while}. Je téměř stejný jako while, ale kontrolní podmínka je umístěna až na konec cyklu. Máme tedy jistotu, že minimálně jednou cyklus vždy proběhne. Syntaxe cyklu je následující:

\begin{verbatim}
int i := 0;

do
start
    i := i + 1; 
end while(i < 4)
\end{verbatim}


\section{Procedury}
Procedura je logický blok kódu, který jednou napíšeme a poté ho můžeme libovolně volat bez toho, abychom ho psali znovu a opakovali se. Proceduru deklarujeme v globálním prostoru, někde nad \textbf{main}.

\subsection{Procedura bez parametrů}
Syntaxe procedura bez parametrů je následující:
\begin{verbatim}
procedure nazevprocedury()
start
    int b := 5;
end

call nazevprocedury();
\end{verbatim}


\subsection{Procedura s parametry}
Procedura může mít také libovolný počet vstupních parametrů (někdy se jim říká argumenty), které píšeme do závorky v její definici. Rozšíříme tedy stávající proceduru o parametr \textbf{int a} a ten potom přidáme s konkrétní hodnotou do volání procedury:
\begin{verbatim}
procedure nazevprocedury(int a)
start
    int b := 5 + a;
end

call nazevprocedury(3);
\end{verbatim}

\subsection{Procedura s návratovou hodnotou}
Procedura může dále vracet nějakou hodnotu, správně by se ji mělo říkat funkce, ale z historických důvodů a zachování kompatibility nazýváme procedura.  Procedura může vracet právě jednu hodnotu pomocí příkazu \textbf{return}. Syntaxe je následující:
\begin{verbatim}
procedure nazevprocedury(int a)
start
    int b := 5 + a;
end
return b;

int y;
call nazevprocedury(3)(y);
\end{verbatim}




\chapter{Implementace}
Pro implementaci překladače byl vybrán jazyk \textbf{Java} a nástroj pro tvorbu syntaktických analyzátorů, kompilátorů a překladačů gramatiky \textbf{ANTLR}. \textbf{ANTLR 4.7} byl vybrán pro jeho snadnou a rychlou použitelnost.
\\\\
Popis adresářové struktury:
\begin{itemize}[noitemsep]
\item [+] \texttt{docs} - dokumentace
\item [+] \texttt{src} - zdrojové kódy
\item [.] \texttt{FJPLexer.g4} - obsahuje tokeny
\item [.] \texttt{FJPParser.g4} - obrahuje gramatiku
\end{itemize}


\chapter{Vzorové programy a generované instrukce}

\section{Program: Podmínky}
Zdrojový kód:
\begin{verbatim}
start
    const KONSTANTA := 5;
    int x := 2;
    int y;
    boolean xx := true;
    boolean yy;

    main
    start
        yy := true;
        [x, y] := [10, 15];
        if(xx && KONSTANTA>3)
        start
            y:= 5;
        end
        else
        start
            y:= 3;
        end
    end
end
\end{verbatim}
\ \\
Instrukce:
\begin{verbatim}
0	JMP 0,1
1	INT 0,7
2	LIT 0,2
3	STO 0,3
4	LIT 0,0
5	STO 0,4
6	LIT 0,1
7	STO 0,5
8	LIT 0,0
9	STO 0,6
10	JMP 0,11
11	INT 0,0
12	LIT 0,1
13	STO 1,7
14	LIT 0,10
15	LIT 0,15
16	STO 1,5
17	STO 1,4
18	LOD 1,6
19	LIT 0,5
20	OPR 0,2
21	LIT 0,2
22	OPR 0,8
23	LIT 0,3
24	OPR 0,12
25	LIT 0,1
26	OPR 0,11
27	JMC 0,31
28	LIT 0,5
29	STO 1,5
30	JMP 0,33
31	LIT 0,3
32	STO 1,5
33	RET 0,0
\end{verbatim}


\section{Program: Cykly}
Zdrojový kód:
\begin{verbatim}
start
    const KONSTANTA := 5;
    int x := 2;

    main
    start
        int b;
        int a := 3;

        while (a = 3)
        start
            a := a + 1;
        end

        do
        start
            x := x * ((2 + a) * 3);
            a, b := a + 1;
        end while (!(a < KONSTANTA))
    end
end
\end{verbatim}
\ \\
Instrukce:
\begin{verbatim}
0	JMP 0,1
1	INT 0,4
2	LIT 0,2
3	STO 0,3
4	JMP 0,5
5	INT 0,2
6	LIT 0,0
7	STO 0,3
8	LIT 0,3
9	STO 0,4
10	LOD 0,4
11	LIT 0,3
12	OPR 0,8
13	LIT 0,1
14	OPR 0,11
15	JMC 0,21
16	LOD 0,4
17	LIT 0,1
18	OPR 0,2
19	STO 0,4
20	JMP 0,10
21	LOD 1,4
22	LIT 0,2
23	LOD 0,4
24	OPR 0,2
25	LIT 0,3
26	OPR 0,4
27	OPR 0,4
28	STO 1,4
29	LOD 0,4
30	LIT 0,1
31	OPR 0,2
32	STO 0,4
33	LOD 0,4
34	STO 0,3
35	LOD 0,4
36	LIT 0,5
37	OPR 0,10
38	LIT 0,0
39	OPR 0,8
40	LIT 0,1
41	OPR 0,10
42	JMC 0,21
43	RET 0,0
\end{verbatim}


\section{Program: Procedury}
Zdrojový kód:
\begin{verbatim}
start
    procedure secti(int a, int b)
    start
        int c;
        c := a + b;
    end
    return c;

    main
    start
        int y;

        call secti(3, 4)(y);
    end
end
\end{verbatim}
\ \\
Instrukce:
\begin{verbatim}
0	JMP 0,1
1	INT 0,3
2	JMP 0,18
3	INT 0,5
4	LIT 0,3
5	STO 0,3
6	LIT 0,4
7	STO 0,4
8	INT 0,1
9	LIT 0,0
10	STO 0,5
11	LOD 0,3
12	LOD 0,4
13	OPR 0,2
14	STO 0,5
15	LOD 0,5
16	STO 1,4
17	RET 0,0
18	INT 0,1
19	LIT 0,0
20	STO 0,3
21	CAL 1,3
22	RET 0,0
\end{verbatim}


\chapter{Závěr}
V semestrální práci byl vytvořen překladač do instrukcí \textbf{PL/0}. Splnily jsme všechny body, které jsme si určily v zadání.
Syntaxe je originálním mixem syntaxe \textbf{C} a \textbf{Pascalu}.
Zdrojové kódy jsou k dispozici na githubu: \url{https://github.com/Ahantuon/fjp} 


\end{document}